\documentclass{ctexart}

\bibliographystyle{plain}
\usepackage{graphicx}
\usepackage{amsmath}
\usepackage{natbib}
\usepackage{url}

\title{作业三: 关于我的操作环境}


\author{王铭恩\\信计3210103777}


\begin{document}

\maketitle

在这门数学软件课程中,我开始使用linux系统,并且在可以预计的将来,我将会大范围使用这个工作环境进行学习。因此,我将在这里说明有关自己的linux环境的一些信息。\cite{mathsoft}
\section{Linux 发行版名称以及版本号}
发行版本:Ubuntu\cite{linux};
    版本号:22.04
\section{调整,安装与额外配置}
首先进行换源操作,将海外源换成清华源,以此来提高下载速度;\\
    其次下载synaptic从而可以进行后续的安装软件操作;
    然后进行输入法调整,使用fcitx(通过synaptic下载)输入方式,同时下载google输入法,然后在重新登入后增加中文输入法;
    而为了数学工作,数学软件需要很多:\\
\verb|g++|;\verb|gcc|;\verb|make|;\verb|cmake|;\verb|automake|;\\
\verb|emacs|;\verb|vim|;\verb|gedit|;\\
\verb|texlive|;\verb|doxygen|;\verb|doxymacs|;\\
\verb|libboost|;\verb|trilinos|;\verb|dx|;\verb|git|;\verb|ssh|;\verb|vnc|;\verb|x11vnc|

\section{工作规划}
\subsection{使用场合}
完成作业文档的撰写与提交;\\
    数学课程结业论文的书写以及提交;\\
    接收课程文件与信息;\\
    进行数学图像的绘制,从而研究其规律以及完成绘图作业;\\
    进行数学科学计算;\\
\subsection{工作环境适应性}
就目前工作环境而言,在可预测的未来,大概率无法满足各种需求:一方面在配置方面会有越来越高的要求,应而需要在用到的时候要通过\verb|get-apt install|或者软件包下载;另一方面,在个人操作方面,在未来使用中难免会遇到新的还未掌握的操作,这时候需要通过阅读工具书或在一些计算机社区中学习与求助,从而学习掌握新的操作。


\section{稳定与安全}
作为虚拟机用户,我会经常备份虚拟机文件,以防意外发生使得虚拟机崩溃导致的悲剧。同时,重要的学习资料与文献资料,我会在云盘中备份(可以有复数个途径),防止由于意外事件导致的文件丢失。在需要多目录i下放置文件时,我不会使用简单的复制粘贴,而是使用链接,防止因为分不清文件而产生失误。

\bibliography{article}

\end{document}

